\documentclass[12pt, letterpaper]{article}
\usepackage[utf8]{inputenc}

\usepackage{listings}
\usepackage{color}
\definecolor{lightgray}{rgb}{.9,.9,.9}
\definecolor{gray}{rgb}{.4,.4,.4}
\definecolor{purple}{rgb}{0.65, 0.12, 0.82}

\lstdefinelanguage{JavaScript}{
  keywords={typeof, new, true, false, catch, function, return, null, catch, switch, var, if, in, while, do, else, case, break},
  keywordstyle=\color{blue}\bfseries,
  ndkeywords={class, export, boolean, throw, implements, import, this},
  ndkeywordstyle=\color{darkgray}\bfseries,
  identifierstyle=\color{black},
  sensitive=false,
  comment=[l]{//},
  morecomment=[s]{/*}{*/},
  commentstyle=\color{purple}\ttfamily,
  stringstyle=\color{red}\ttfamily,
  morestring=[b]',
  morestring=[b]"
}

\lstset{
   language=JavaScript,
   backgroundcolor=\color{lightgray},
   extendedchars=true,
   basicstyle=\footnotesize\ttfamily,
   showstringspaces=false,
   showspaces=false,
   numbers=left,
   numberstyle=\footnotesize,
   numbersep=9pt,
   tabsize=2,
   breaklines=true,
   showtabs=false,
   captionpos=b
}

\title{Else Statement}
\author{Justin Besteman}
\date{December 30, 2016}

\begin{document}


\maketitle


\section*{Synopsis}

Statement that can be attached to an If statement. 


\section*{Why}

Allows the coder to test a condition if the condition is not met it will excuted the else statement. \\
Instead of a coder writing two if statement to try the condition is true or not, the coder can just use a if else statement. \\
I.E
\begin{lstlisting}
var test = true;

if (test == true) {

	// If test is true...
	// Do this

} // End of If Statement

if (test == false) {

	// If test is false
	// Do this

} // End of If Statement

// A coder can write this
// This will achieve the same effect 
// With less code

if (test == true) {

	// If test is true
	// Do this

} else {

	// If false
	// Do this

} // End of If Else Statement
\end{lstlisting}

\section*{Syntax}

Else statement must have an If statement attached to it. \\
Does not take conditions to test for the condition being tested is the if statement first. \\ 
If the if statement conditions are not met it will excute the else statement. \\
Else statement is block level code so needs the curly braces.

\section*{Advanced}



\section*{Examples}

\begin{lstlisting}

var bored = true;

if (bored == true) {

	console.log("Watch Firefly");

} else {

	console.log("Don't watch Firefly");
	// Bad example because I would still watch Firefly

} // End of If Else Statement

\end{lstlisting}

\end{document}