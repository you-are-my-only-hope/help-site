\documentclass[12pt, letterpaper]{article}
\usepackage[utf8]{inputenc}

\usepackage{listings}
\usepackage{color}
\definecolor{lightgray}{rgb}{.9,.9,.9}
\definecolor{gray}{rgb}{.4,.4,.4}
\definecolor{purple}{rgb}{0.65, 0.12, 0.82}

\lstdefinelanguage{JavaScript}{
  keywords={typeof, new, true, false, catch, function, return, null, catch, switch, var, if, in, while, do, else, case, break},
  keywordstyle=\color{blue}\bfseries,
  ndkeywords={class, export, boolean, throw, implements, import, this},
  ndkeywordstyle=\color{darkgray}\bfseries,
  identifierstyle=\color{black},
  sensitive=false,
  comment=[l]{//},
  morecomment=[s]{/*}{*/},
  commentstyle=\color{purple}\ttfamily,
  stringstyle=\color{red}\ttfamily,
  morestring=[b]',
  morestring=[b]"
}

\lstset{
   language=JavaScript,
   backgroundcolor=\color{lightgray},
   extendedchars=true,
   basicstyle=\footnotesize\ttfamily,
   showstringspaces=false,
   showspaces=false,
   numbers=left,
   numberstyle=\footnotesize,
   numbersep=9pt,
   tabsize=2,
   breaklines=true,
   showtabs=false,
   captionpos=b
}

\title{Else Statement}
\author{Justin Besteman}
\date{December 30, 2016}

\begin{document}


\maketitle


\section*{Synopsis}

Statement that can be attached to an If statement. 


\section*{Why}

Allows the coder to run alternate code when the test condition in an if statement is not met. Instead of a coder writing two if statements to test whether the condition is true or not, the coder can just use an if else statement. \\
E.G. \\
\begin{lstlisting}
var test = true;

if (test == true) {

	// If test is true...
	// Do this

} // End of If Statement

if (test == false) {

	// If test is false
	// Do this

} // End of If Statement

// Instead, a coder can write the following
// which will achieve the same result 
// with less code

if (test == true) {

	// If test is true
	// Do this

} else {

	// If false
	// Do this

} // End of If Else Statement
\end{lstlisting}

\section*{Syntax}

Else statement must have an If statement attached to it. It does not take any conditions to test because the attached if statement does that job. If the if statement conditions are not met then the else statement is executed instead. Else statement is block level code so it uses curly braces.

\section*{Advanced}



\section*{Examples}

\begin{lstlisting}

var bored = true;

if (bored == true) {

	console.log("Watch Firefly");

} else {

	console.log("Don't watch Firefly");
	// Bad example because I would still watch Firefly

} // End of If Else Statement

\end{lstlisting}

\end{document}