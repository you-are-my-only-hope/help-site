\documentclass[12pt, letterpaper]{article}
\usepackage[utf8]{inputenc}

\usepackage{listings}
\usepackage{color}
\definecolor{lightgray}{rgb}{.9,.9,.9}
\definecolor{gray}{rgb}{.4,.4,.4}
\definecolor{purple}{rgb}{0.65, 0.12, 0.82}

\lstdefinelanguage{JavaScript}{
  keywords={typeof, new, true, false, catch, function, return, null, catch, switch, var, if, in, while, do, else, case, break},
  keywordstyle=\color{blue}\bfseries,
  ndkeywords={class, export, boolean, throw, implements, import, this},
  ndkeywordstyle=\color{darkgray}\bfseries,
  identifierstyle=\color{black},
  sensitive=false,
  comment=[l]{//},
  morecomment=[s]{/*}{*/},
  commentstyle=\color{purple}\ttfamily,
  stringstyle=\color{red}\ttfamily,
  morestring=[b]',
  morestring=[b]"
}

\lstset{
   language=JavaScript,
   backgroundcolor=\color{lightgray},
   extendedchars=true,
   basicstyle=\footnotesize\ttfamily,
   showstringspaces=false,
   showspaces=false,
   numbers=left,
   numberstyle=\footnotesize,
   numbersep=9pt,
   tabsize=2,
   breaklines=true,
   showtabs=false,
   captionpos=b
}


\title{Variables}
\author{Justin Besteman}
\date{December 30, 2016}

\begin{document}


\maketitle


\section*{Synopsis}

	 Variables are the foundation of programming and it goes way beyond JavaScript. Variables exist so that a programmer can dynamically write code that can be assigned to certain variables that can be changed during the process of coding. 

\section*{Why}

Variables are used by programmers to make declarations that equal one another.  For example, your car miles per gallon equals 35. The programmer can take this information and declare that MPG equals 35.  This allows the coder to actually manipulate this data and to be able to use it to calculate certain things. 

\section*{Syntax}

The two main components of declaring variables are the equal symbol and the keyword vars. Vars stands for variable. \\
\\
E.G.
\begin{lstlisting}
var MPG = 35;
\end{lstlisting}
Now the value of MPG throughout the program, unless changed by the coder, is equal to 35.

\section*{Advanced}

There are three major types of variables: integer, string, and boolean. Integers are what we have seen in the Syntax section. Strings are words or letters enclosed by quotes.\\ \\
E.G.
\begin{lstlisting}
var day = "Friday";
\end{lstlisting}
Booleans are very powerful variables that equate to True or False. They can only be True or False. No maybes in computers!

\begin{lstlisting}
var gym = False;
var happy = True;
var female = true;
\end{lstlisting}


\section*{Examples}

\begin{lstlisting}
var hello = "Hello World"
var height = 5.8;
var adult = true;
\end{lstlisting}


\end{document}