\documentclass[12pt, letterpaper]{article}
\usepackage[utf8]{inputenc}

\usepackage{listings}
\usepackage{color}
\definecolor{lightgray}{rgb}{.9,.9,.9}
\definecolor{gray}{rgb}{.4,.4,.4}
\definecolor{purple}{rgb}{0.65, 0.12, 0.82}

\lstdefinelanguage{JavaScript}{
  keywords={typeof, new, true, false, catch, function, return, null, catch, switch, var, if, in, while, do, else, case, break},
  keywordstyle=\color{blue}\bfseries,
  ndkeywords={class, export, boolean, throw, implements, import, this},
  ndkeywordstyle=\color{darkgray}\bfseries,
  identifierstyle=\color{black},
  sensitive=false,
  comment=[l]{//},
  morecomment=[s]{/*}{*/},
  commentstyle=\color{purple}\ttfamily,
  stringstyle=\color{red}\ttfamily,
  morestring=[b]',
  morestring=[b]"
}

\lstset{
   language=JavaScript,
   backgroundcolor=\color{lightgray},
   extendedchars=true,
   basicstyle=\footnotesize\ttfamily,
   showstringspaces=false,
   showspaces=false,
   numbers=left,
   numberstyle=\footnotesize,
   numbersep=9pt,
   tabsize=2,
   breaklines=true,
   showtabs=false,
   captionpos=b
}

\usepackage{hyperref}
\hypersetup{
    colorlinks=true,
    linkcolor=blue,
    filecolor=magenta,      
    urlcolor=cyan,
}

\title{Array}
\author{Justin Besteman}
\date{Reviewed by Ali Zaidi}
\begin{document}


\maketitle


\section*{Synopsis}

Arrays are a foundation of programming and are found in most major programming languages.

\section*{Why}

What a coder can do with arrays, they can do with variables. So why use arrays? It's because arrays allow us to make lists of variables which can be vastly easier to use and code with. Variables are good to use when you have a few things to store that don't warrant a list of variables. With a car mpg, that would be just a single variable but how about the names of cars? Do you want to sit there and type var car1 = "", var car2 = "" and so on? \\


\section*{Syntax}

Just like when I think of my wife I think "glorious," when you thnk of arrays you should think "square brackets."
\begin{lstlisting}
var array = [];

// This array is empty and holds nothing. 
\end{lstlisting}
The content of the array is always (always!!!) enclosed in [ ];\\ \\
An array can hold all types of data that a variable can. I.E., arrays can hold integers, floats (decimals), strings, booleans, and even variables themselves. A programmer adds the comma, , ,(yes that is a comma enclosed by commas) to delimit the data inside the array.\\
\begin{lstlisting}
var number = [1 , 2 , 3 , 4 , 5];
\end{lstlisting}
To access the data inside the array, a coder uses: 
\begin{lstlisting}
var number = [1 , 2 , 3 , 4 , 5];

number[0];
\end{lstlisting}
A programmer can put any number, or variable representing a number, inside the square brackets to access what is stored at that point in the array. 
\section*{Advanced}

Very Important!!! In 99.7\% of coding languages, an array starts counting at index 0 and not at index 1!
\begin{lstlisting}
var greet = ["Hi","Bye","Hello","Goodbye"];
\end{lstlisting}
If a coder wants to access the data of ``Hi", it is \textbf{not index one}, it is index zero! The technical word for the location number that stores the data is \textbf{index}.

\section*{Examples}

\begin{lstlisting}

// --------------------- Example 1 -----------------

var car = ["chevy" , "ford", "honda"];

// --------------------- Example 2 -----------------

var test = [true , true , false , true, false, false];

// --------------------- Example 3 -----------------

//   index    0     1      2        3
var greet = ["Hi","Bye","Hello","Goodbye"];

greet[0]; // "Hi"
greet[1]; // "Bye"
greet[2]; // "Hello"
greet[3]; // "Goodbye"

// --------------------- Example 5 -----------------

var number = 3;

greet[number]; // "Hello"

// --------------------- Example 6 -----------------

var number = 3;

var greeting = greet[number]; // "Hello"

// Now greeting is a variable that is set equal to 
// the value at greet index 3 which is "Hello"


\end{lstlisting}

\section*{Further Reading}

\href{https://developer.mozilla.org/en-US/docs/Glossary/array}{MDN - Arrays}\\
\\
\href{https://developer.mozilla.org/en-US/docs/Learn/JavaScript/First_steps/Arrays}{MDN - Arrays Advance}\\
\\
\href{https://www.tutorialspoint.com/javascript/javascript_arrays_object.htm}{Tutorials Points - Arrays}\\
\\
\href{http://www.w3schools.com/js/js_arrays.asp}{w3schools - Array}

\end{document}