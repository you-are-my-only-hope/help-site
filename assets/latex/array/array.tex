\documentclass[12pt, letterpaper]{article}
\usepackage[utf8]{inputenc}

\usepackage{listings}
\usepackage{color}
\definecolor{lightgray}{rgb}{.9,.9,.9}
\definecolor{gray}{rgb}{.4,.4,.4}
\definecolor{purple}{rgb}{0.65, 0.12, 0.82}

\lstdefinelanguage{JavaScript}{
  keywords={typeof, new, true, false, catch, function, return, null, catch, switch, var, if, in, while, do, else, case, break},
  keywordstyle=\color{blue}\bfseries,
  ndkeywords={class, export, boolean, throw, implements, import, this},
  ndkeywordstyle=\color{darkgray}\bfseries,
  identifierstyle=\color{black},
  sensitive=false,
  comment=[l]{//},
  morecomment=[s]{/*}{*/},
  commentstyle=\color{purple}\ttfamily,
  stringstyle=\color{red}\ttfamily,
  morestring=[b]',
  morestring=[b]"
}

\lstset{
   language=JavaScript,
   backgroundcolor=\color{lightgray},
   extendedchars=true,
   basicstyle=\footnotesize\ttfamily,
   showstringspaces=false,
   showspaces=false,
   numbers=left,
   numberstyle=\footnotesize,
   numbersep=9pt,
   tabsize=2,
   breaklines=true,
   showtabs=false,
   captionpos=b
}

\title{Array}
\author{Justin Besteman}
\date{December 30, 2016}

\begin{document}


\maketitle


\section*{Synopsis}

Arrays are a foundation of programming and are found in most major programming languages.

\section*{Why}

What a coder can do with arrays, they can do with variables. So why use learn arrays? \\
Because arrays allow how to make list of variables which can be vastly easier to use and code with.\\
Variable are good to use when you have a few things to store that don't warrent list of variable. \\
With a car mpg, that would be a just a single variable but how about the names of cars. \\
Do you want to sit there and type var car1 = "", var car2 = "" and so? \\


\section*{Syntax}

Just like when I think of my wife, I think "glorious" and when you thnk of arrays think "square brackets"
\begin{lstlisting}
var array = [];

// This array is empty and holds nothing. 
\end{lstlisting}
The content of the array is always (always!!!) enclosed in [ ];\\
A array can hold all types of data that a variable can. \\
So integer, float(decimals), strings, booleans, and it can hold variables themselves\\
A coder adds the comma, , ,(yes that is a comma enclosed by commas) to delimit the data inside the array\\
\begin{lstlisting}
var number = [1 , 2 , 3 , 4 , 5];
\end{lstlisting}
To access the data inside the array, a coder usings 
\begin{lstlisting}
var number = [1 , 2 , 3 , 4 , 5];

number[0];
\end{lstlisting}
So a coder can put any number or variable of a number inside the square brackets to access what is stored at that point in the array. 
\section*{Advanced}

Very Important!!! In 99.7\% of coding languages, an array starts counting at 0 and not a 1
\begin{lstlisting}
var greet = ["Hi","Bye","Hello","Goodbye"];
\end{lstlisting}
So if a coder wants to access the data of ``Hi", it \textbf{not one}, it is zero\\
The technical word for the number that stores the data is \textbf{index}

\section*{Examples}

\begin{lstlisting}

// --------------------- Example 1 -----------------

var car = ["chevy" , "ford", "honda"];

// --------------------- Example 2 -----------------

var test = [true , true , false , true, false, false];

// --------------------- Example 3 -----------------

//   index    0     1      2        3
var greet = ["Hi","Bye","Hello","Goodbye"];

greet[0]; // "Hi"
greet[1]; // "Bye"
greet[2]; // "Hello"
greet[3]; // "Goodbye"

// --------------------- Example 5 -----------------

var number = 3;

greet[number]; // "Hello"

// --------------------- Example 6 -----------------

var number = 3;

var greeting = greet[number]; // "Hello"

// Now greeting is a variable that is set equal to greet index 3 which, 
// Is "Hello"


\end{lstlisting}

\end{document}