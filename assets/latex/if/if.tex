\documentclass[12pt, letterpaper]{article}
\usepackage[utf8]{inputenc}

\usepackage{listings}
\usepackage{color}
\definecolor{lightgray}{rgb}{.9,.9,.9}
\definecolor{gray}{rgb}{.4,.4,.4}
\definecolor{purple}{rgb}{0.65, 0.12, 0.82}

\lstdefinelanguage{JavaScript}{
  keywords={typeof, new, true, false, catch, function, return, null, catch, switch, var, if, in, while, do, else, case, break},
  keywordstyle=\color{blue}\bfseries,
  ndkeywords={class, export, boolean, throw, implements, import, this},
  ndkeywordstyle=\color{darkgray}\bfseries,
  identifierstyle=\color{black},
  sensitive=false,
  comment=[l]{//},
  morecomment=[s]{/*}{*/},
  commentstyle=\color{purple}\ttfamily,
  stringstyle=\color{red}\ttfamily,
  morestring=[b]',
  morestring=[b]"
}

\lstset{
   language=JavaScript,
   backgroundcolor=\color{lightgray},
   extendedchars=true,
   basicstyle=\footnotesize\ttfamily,
   showstringspaces=false,
   showspaces=false,
   numbers=left,
   numberstyle=\footnotesize,
   numbersep=9pt,
   tabsize=2,
   breaklines=true,
   showtabs=false,
   captionpos=b
}

\title{If Statement}
\author{Justin Besteman}
\date{December 30, 2016}

\begin{document}


\maketitle


\section*{Synopsis}

Foundation of all logic base programming. It allows a coder to test conditions for the flow of the program.

\section*{Why}

Coders will use this statement a lot in coding. The if statement is the key piece of equipment in the coder's toolbox. Think about wanting to test whether something passes a certian conidition or if it doesn't.

\section*{Syntax}

Start with the keyword ``if" followed by parenthesis ``()" then curly braces $\left\{\right\}$ \\ \\
If statement is code that we call block level code meaning the if statement takes up more then one line of code. When we declared variables in the past, the computer just read and executed that one line. However, with block level code the computer will read the entire code inside the curly braces $\left\{\right\}$ \\ \\
Inside the parentheses, programmers put the condition that they want to test. Then, based on the whether that condition is true or false, the code inside the curly braces is executed or bypassed. \\ \\
E.G.\\
\begin{lstlisting}
if (5 == 5) {

	console.log("They are equal");

} // End of If Statement

if (1 > 5) {

	// This code will not run because the condition is not true
	console.log("True. 1 is greater then 5");

} // End of If Statement
\end{lstlisting}
Notice the  ``==", a coder will use the double equals to ask the computer if variable or condition is true 

\section*{Advanced}

The conditions being tested do not have to just be integers, but instead can be variables like: \\

\begin{lstlisting}
var number = 5;

if (number == 5) {

	console.log("The variable number is equal to five");

} // End of If Statement

var test = 6;

if (number == test){

	// This code will not run
	console.log("Variable number and test are equal");

} // End of If Statement
\end{lstlisting}
The default of an If statement is true but a coder can change the default to false by using a ``!". In this manner, the code inside the parenthesis then has to be false for the code inside the curly braces to run. \\

\begin{lstlisting}
if !(6 == 5) {

	console.log("They are not equal");

} // End of If Statement
\end{lstlisting}
A coder can test multiple conditions in the parenthesis but using the \&\& or  $\Vert$. \&\& means ``and" so both conditions have to be true for the test to evaluate as true. $\Vert$ means ``or" so one (or both) of the conditions have to be true for the test to evaluate as true.

\begin{lstlisting}

var number = 5;

if (number == 5 && number > 1) {

	console.log("The number is equal to 5 and it is greater than 5");

} // End of If Statement


if (number == 5 || number < 1) {

	console.log("The number is equal to 5 or it is less than 5");

} // End of If Statement

\end{lstlisting}
\section*{Examples}

\begin{lstlisting}


// ----------------- Example 1 ------------------

var hungry = true;

if (hungry == true) {

	console.log("Let's go eat");

} // End of If Statement



// ----------------- Example 2 ------------------

// Since the if statement default is true,
// a coder can write this

if (hungry) {

	console.log("Let's go eat");

} // End of If Statement



// ----------------- Example 3 ------------------

// Also a coder can do this

var hungry = false;

if (!hungry) {

	console.log("I am full");

} // End of If Statement



// ----------------- Example 4 ------------------

var num = [1 , 2 , 3 , 4 , 5 , 6 , 7 , 8 , 9 , 10];

for (var i = 0 ; i < 9 ; i++) {

	if (num[i] == 5) {

		console.log("5 is in the array of num");
	
	}

} // End of For Loop I

\end{lstlisting}


\end{document}