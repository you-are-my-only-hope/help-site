\documentclass[12pt, letterpaper]{article}
\usepackage[utf8]{inputenc}

\usepackage{listings}
\usepackage{color}
\definecolor{lightgray}{rgb}{.9,.9,.9}
\definecolor{gray}{rgb}{.4,.4,.4}
\definecolor{purple}{rgb}{0.65, 0.12, 0.82}

\lstdefinelanguage{JavaScript}{
  keywords={typeof, new, true, false, catch, function, return, null, catch, switch, var, if, in, while, do, else, case, break},
  keywordstyle=\color{blue}\bfseries,
  ndkeywords={class, export, boolean, throw, implements, import, this},
  ndkeywordstyle=\color{darkgray}\bfseries,
  identifierstyle=\color{black},
  sensitive=false,
  comment=[l]{//},
  morecomment=[s]{/*}{*/},
  commentstyle=\color{purple}\ttfamily,
  stringstyle=\color{red}\ttfamily,
  morestring=[b]',
  morestring=[b]"
}

\lstset{
   language=JavaScript,
   backgroundcolor=\color{lightgray},
   extendedchars=true,
   basicstyle=\footnotesize\ttfamily,
   showstringspaces=false,
   showspaces=false,
   numbers=left,
   numberstyle=\footnotesize,
   numbersep=9pt,
   tabsize=2,
   breaklines=true,
   showtabs=false,
   captionpos=b
}

\usepackage{hyperref}
\hypersetup{
    colorlinks=true,
    linkcolor=blue,
    filecolor=magenta,      
    urlcolor=cyan,
}

\title{Else If Statement}
\author{Justin Besteman}
\date{Reviewed by Ali Zaidi}

\begin{document}


\maketitle


\section*{Synopsis}

Else If statements allows programmers to test multiple conditions.

\section*{Why}

We have seen what the If statement and the Else statement can do. They do a great job of testing conditions that are binary. I.E., a person is of legal drinking age or not, a person is either male or female, 5 is equal to 5 or not, and so on. The If and Else statement are great for those situations where if one is true then the other is false and vice versa. \\ \\
However, sometimes conditions are not binary but poly (many). I.E., the traffic light is green / yellow / red, a person can go left / right / straight / turn around, and so on. \\

\section*{Syntax}

Else if has to have an If statement attached to it. It is block level code and very similar to the if statement. After the first if statement, a coder will write ``else if" and another condition to test.

\section*{Advanced}

A coder can use as many ``else if" statements as needed but if there are many conditions to test then I would suggest implementing another coding technique called a switch statement which we will cover later.

\section*{Examples}

\begin{lstlisting}

// ------------------- Example 1 ---------------------

var traffic_light = "yellow";

if (traffic_light == "grean") {

	console.log("Go");

} else if (traffic_light == "yellow") {

	console.log("Speed up to make the light or stop");

} else if (traffic_light == "red") {

	console.log("Please stop! Think of the Children");

} // End of the If Else If Statement


// ------------------- Example 2 ---------------------

// Notice you can change this statement like this

var traffic_light = "yellow";

if (traffic_light == "grean") {

	console.log("Go");

} else if (traffic_light == "yellow") {

	console.log("Speed up to make the light or stop");

} else {

	console.log("Please stop! Think of the Children");

} // End of the If Else If Statement

// Both of example 1 and 2 will do the same,
// If the light is not green or yellow it has to be red

// ------------------- Example 3 ---------------------

var user_pick = 3;

if (user_pick == 1) {

	console.log("User picks 1");

} else if (user_pick == 2) {

	console.log("User picks 2");

} else if (user_pick == 3) {

	console.log("User picks 3");

} else if (user_pick == 4) {

	console.log("User picks 4");

} else {

	console.log("User picks 5");	

}
\end{lstlisting}

\section*{Further Reading}

\href{https://developer.mozilla.org/en-US/docs/Web/JavaScript/Reference/Statements/if...else}{MDN - If Else}\\
\\
\href{http://www.w3schools.com/js/js_if_else.asp}{w3schools - If Else}\\
\\
\href{https://www.tutorialspoint.com/javascript/javascript_ifelse.htm}{Tutorials Points - If Else}

\end{document}