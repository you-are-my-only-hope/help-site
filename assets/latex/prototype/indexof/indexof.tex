\documentclass[12pt, letterpaper]{article}
\usepackage[utf8]{inputenc}

\usepackage{listings}
\usepackage{color}
\definecolor{lightgray}{rgb}{.9,.9,.9}
\definecolor{gray}{rgb}{.4,.4,.4}
\definecolor{purple}{rgb}{0.65, 0.12, 0.82}

\lstdefinelanguage{JavaScript}{
  keywords={typeof, new, true, false, catch, function, return, null, catch, switch, var, if, in, while, do, else, case, break,indexOf},
  keywordstyle=\color{blue}\bfseries,
  ndkeywords={class, export, boolean, throw, implements, import, this},
  ndkeywordstyle=\color{darkgray}\bfseries,
  identifierstyle=\color{black},
  sensitive=false,
  comment=[l]{//},
  morecomment=[s]{/*}{*/},
  commentstyle=\color{purple}\ttfamily,
  stringstyle=\color{red}\ttfamily,
  morestring=[b]',
  morestring=[b]"
}

\lstset{
   language=JavaScript,
   backgroundcolor=\color{lightgray},
   extendedchars=true,
   basicstyle=\footnotesize\ttfamily,
   showstringspaces=false,
   showspaces=false,
   numbers=left,
   numberstyle=\footnotesize,
   numbersep=9pt,
   tabsize=2,
   breaklines=true,
   showtabs=false,
   captionpos=b
}


\usepackage{hyperref}
\hypersetup{
    colorlinks=true,
    linkcolor=blue,
    filecolor=magenta,      
    urlcolor=cyan,
}

\title{IndexOf}
\author{Justin Besteman}
\date{Reviewed by Ali Zaidi}

\begin{document}



\maketitle


\section*{Synopsis}

A powerful prototype for searching an array for a certain element.

\section*{Why}

When working with arrays, there are going to be times when a coder needs to know if a certain element or item is in the array. For example, a coder may want to have a search field where a user may enter a search term to check whether that item exists or not.

\section*{Syntax}

The syntax is very simple. The name of the array and indexOf() connected by a dot or period. Inside the parentheses you list the item you wish to search for within the array.

\section*{Advanced}

The return value of indexOf is the index number identifying the location of the item within the array. If the item is not found in the array then the returned value is -1. If the item exists many times within the array then only the index number of the first occurrance within the array is returned.\\

\section*{Examples}

\begin{lstlisting}
// ---------------- Example 1 -----------------

var magic = [1,4,3,2,5,99];

magic.indexOf(4); // 1
magic.indexOf(99); // 5
magic.indexOf(6); // -1

// ---------------- Example 2 -----------------

var meatloaf = ["Dashboard", "two-thirds", "Dry Eye"];

var song = meatloaf.indexOf("Dry Eye");

// song is now 2 

var other_song = meatloaf.indexOf("dashboard");

// other_song is equal to -1 because the search is case sensitive 

// ---------------- Example 3 -----------------   

var dream = ["clouds", "daydreaming", "sunshine", "toothfairy"];

var check = dream.indexOf("daydreaming");

if (check != -1) {

	console.log("We are daydreaming now");

} // End of If Statement

// ---------------- Example 4 -----------------

var dream = ["clouds", "daydreaming", "sunshine", "toothfairy"];

var check = dream.indexOf("daydreaming");

if (check >= 0) {

	console.log("We are daydreaming now");

} // End of If Statement

// ---------------- Example 5 -----------------

var dream = ["clouds", "daydreaming", "sunshine", "toothfairy"];

var check = "daydreaming";

if (dream.indexOf(check) >= 0) {

	console.log("We are daydreaming now");

} // End of If Statement

// ---------------- Example 6 -----------------

var numbers = [1,2,2,3,3,3];

numbers.indexOf(2); // 1
numbers.indexOf(3); // 3

// Remember: indexOf will only return the index
// of the first occurance of your search item. It
// WILL NOT return multiple indexes


\end{lstlisting}

\section*{Further Reading}

\href{https://developer.mozilla.org/en-US/docs/Web/JavaScript/Reference/Global_Objects/Array/indexOf}{MDN - indexOf}\\
\\
\href{http://www.w3schools.com/jsref/jsref_indexof_array.asp}{w3schools - indexOf}\\

\end{document}
