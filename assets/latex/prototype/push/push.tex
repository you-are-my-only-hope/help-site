\documentclass[12pt, letterpaper]{article}
\usepackage[utf8]{inputenc}

\usepackage{listings}
\usepackage{color}
\definecolor{lightgray}{rgb}{.9,.9,.9}
\definecolor{gray}{rgb}{.4,.4,.4}
\definecolor{purple}{rgb}{0.65, 0.12, 0.82}

\lstdefinelanguage{JavaScript}{
  keywords={typeof, new, true, false, catch, function, return, null, catch, switch, var, if, in, while, do, else, case, break},
  keywordstyle=\color{blue}\bfseries,
  ndkeywords={class, export, boolean, throw, implements, import, this},
  ndkeywordstyle=\color{darkgray}\bfseries,
  identifierstyle=\color{black},
  sensitive=false,
  comment=[l]{//},
  morecomment=[s]{/*}{*/},
  commentstyle=\color{purple}\ttfamily,
  stringstyle=\color{red}\ttfamily,
  morestring=[b]',
  morestring=[b]"
}

\lstset{
   language=JavaScript,
   backgroundcolor=\color{lightgray},
   extendedchars=true,
   basicstyle=\footnotesize\ttfamily,
   showstringspaces=false,
   showspaces=false,
   numbers=left,
   numberstyle=\footnotesize,
   numbersep=9pt,
   tabsize=2,
   breaklines=true,
   showtabs=false,
   captionpos=b
}


\usepackage{hyperref}
\hypersetup{
    colorlinks=true,
    linkcolor=blue,
    filecolor=magenta,      
    urlcolor=cyan,
}

\title{Push}
\author{Justin Besteman}
\date{Reviewed by Ali Zaidi}

\begin{document}

\maketitle

\section*{Synopsis}

This will allow a coder to add items to a declared array.

\section*{Why}

When a coder declares an array, that coder might need to update the array with other items. \\
So if a coder writes a list of movies that he/she owns like\\
\begin{lstlisting}
var movie = ["The Thing","Firefly","Rush Hour", "Friday"];
\end{lstlisting}
What happens if the coder buys another movie the coder would want a way to add the movie to the array.
\section*{Syntax}
To do this you need to state the name of the array\\
Then you need to write push connected with a ``." \\
After the push, you write parenthesis ``()"\\
Inside the parenthesis, you place want you want to be added to the array
\section*{Advanced}

The element that is being added will be added to the end of the array. \\
So think of reloading bullets in to a magazine, you push the bullets on the top of the magazine.\\
This will change the result of .length on an array. 

\section*{Examples}

\begin{lstlisting}
// ------------------ Example 1 -----------------------

var movie = ["The Thing","Firefly","Rush Hour", "Friday"];

movie.push("Angel"); 
// ["The Thing","Firefly","Rush Hour", "Friday", "Angel"]

// ------------------ Example 2 -----------------------

var movie = ["The Thing","Firefly","Rush Hour", "Friday"];

movie.push("Angel" , "Fight Club"); 
// ["The Thing","Firefly","Rush Hour", "Friday", "Angel" ,"Fight Club"]

// ------------------ Example 3 -----------------------

var number_array = [];

for (var i = 0; i < 5; i++){

	number_array.push(i);

} // End of For Loop

console.log(number_array);
// 0 , 1 , 2 , 3 , 4

\end{lstlisting}

\section*{Further Reading}

\href{https://developer.mozilla.org/en-US/docs/Web/JavaScript/Reference/Global_Objects/Array/push}{MDN - Push}\\
\\
\href{http://www.w3schools.com/jsref/jsref_push.asp}{w3schools - Push}\\
\\
\href{http://stackoverflow.com/questions/351409/appending-to-array}{StackOverFlow - Push}

\end{document}
