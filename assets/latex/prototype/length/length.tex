\documentclass[12pt, letterpaper]{article}
\usepackage[utf8]{inputenc}

\usepackage{listings}
\usepackage{color}
\definecolor{lightgray}{rgb}{.9,.9,.9}
\definecolor{gray}{rgb}{.4,.4,.4}
\definecolor{purple}{rgb}{0.65, 0.12, 0.82}

\lstdefinelanguage{JavaScript}{
  keywords={typeof, new, true, false, catch, function, return, null, catch, switch, var, if, in, while, do, else, case, break},
  keywordstyle=\color{blue}\bfseries,
  ndkeywords={class, export, boolean, throw, implements, import, this},
  ndkeywordstyle=\color{darkgray}\bfseries,
  identifierstyle=\color{black},
  sensitive=false,
  comment=[l]{//},
  morecomment=[s]{/*}{*/},
  commentstyle=\color{purple}\ttfamily,
  stringstyle=\color{red}\ttfamily,
  morestring=[b]',
  morestring=[b]"
}

\lstset{
   language=JavaScript,
   backgroundcolor=\color{lightgray},
   extendedchars=true,
   basicstyle=\footnotesize\ttfamily,
   showstringspaces=false,
   showspaces=false,
   numbers=left,
   numberstyle=\footnotesize,
   numbersep=9pt,
   tabsize=2,
   breaklines=true,
   showtabs=false,
   captionpos=b
}


\usepackage{hyperref}
\hypersetup{
    colorlinks=true,
    linkcolor=blue,
    filecolor=magenta,      
    urlcolor=cyan,
}

\title{Length}
\author{Justin Besteman}
\date{December 30, 2016}

\begin{document}



\maketitle


\section*{Synopsis}

Length is one of the most used prototype of an array. 

\section*{Why}

Often, when I say often I mean all the time, a coder will want to know the length of an array. This is needed to be enable to 
know how many items/indexs are in the array to loop over or search through. \\
Say that a coder puts the amount of movies they own or a company owns into an array:
\begin{lstlisting}
var movie = ["The Thing","Firefly","Rush Hour", "Friday the 13th"];
\end{lstlisting}
To find out how movies there are, all you need to do is run the prototype length on the array.
\section*{Syntax}
The syntax is very easy. The name of the array and then length
\begin{lstlisting}
movie.length // 4
\end{lstlisting}
\section*{Advanced}

\section*{Examples}

\begin{lstlisting}

// ------------------ Example 1 --------------

var array_numbers = [11,2,1,3,4];

array_numbers.length // 5		

// ------------------ Example 2 --------------

// length is very powerful when you use it,
// with a for loop when you want to loop over it

var greet = ["hi" , "bye" , "hello" , "goodbye"];

var length = greet.length;

for (var i = 0 ; i < length ; i++){

	console.log(greet[i]);

} // End of For Loop

// ------------------ Example 2 --------------

var greet = ["hi" , "bye" , "hello" , "goodbye"];

var length = greet.length;

for (var i = 0 ; i < greet.length ; i++){

	console.log(greet[i]);

} // End of For Loop


// You don't need to make a seperate variable
// If you don't want to


		
\end{lstlisting}

\section*{Further Reading}

\href{https://developer.mozilla.org/en-US/docs/Web/JavaScript/Reference/Global_Objects/Array/length}{MDN - length}

\end{document}
