\documentclass[12pt, letterpaper]{article}
\usepackage[utf8]{inputenc}

\usepackage{listings}
\usepackage{color}
\definecolor{lightgray}{rgb}{.9,.9,.9}
\definecolor{gray}{rgb}{.4,.4,.4}
\definecolor{purple}{rgb}{0.65, 0.12, 0.82}

\lstdefinelanguage{JavaScript}{
  keywords={typeof, new, true, false, catch, function, return, null, catch, switch, var, if, in, while, do, else, case, break},
  keywordstyle=\color{blue}\bfseries,
  ndkeywords={class, export, boolean, throw, implements, import, this},
  ndkeywordstyle=\color{darkgray}\bfseries,
  identifierstyle=\color{black},
  sensitive=false,
  comment=[l]{//},
  morecomment=[s]{/*}{*/},
  commentstyle=\color{purple}\ttfamily,
  stringstyle=\color{red}\ttfamily,
  morestring=[b]',
  morestring=[b]"
}

\lstset{
   language=JavaScript,
   backgroundcolor=\color{lightgray},
   extendedchars=true,
   basicstyle=\footnotesize\ttfamily,
   showstringspaces=false,
   showspaces=false,
   numbers=left,
   numberstyle=\footnotesize,
   numbersep=9pt,
   tabsize=2,
   breaklines=true,
   showtabs=false,
   captionpos=b
}


\usepackage{hyperref}
\hypersetup{
    colorlinks=true,
    linkcolor=blue,
    filecolor=magenta,      
    urlcolor=cyan,
}

\title{Length}
\author{Justin Besteman}
\date{Reviewed by Ali Zaidi}

\begin{document}



\maketitle


\section*{Synopsis}

Length is one of the most often used prototypes of arrays. 

\section*{Why}

Often (and when I say often I mean all the time) a programmer will want to know the length of an array. The length lets the programmer know how many items (indexes) are in the array to loop over or search through. \\ \\
Let's say that a programmer stores the movies they own into an array:
\begin{lstlisting}
var movie = ["The Thing", "Firefly", "Rush Hour", "Friday the 13th"];
\end{lstlisting}
To find out how movies are stored in the array, all they need to do is run the prototype length on the array.
\section*{Syntax}
The syntax is very easy. The name of the array and then length:
\begin{lstlisting}
movie.length // 4
\end{lstlisting}
\section*{Advanced}

\section*{Examples}

\begin{lstlisting}

// ------------------ Example 1 --------------

var array_numbers = [11,2,1,3,4];

array_numbers.length // 5		

// ------------------ Example 2 --------------

// length is very powerful when you use it
// with a for loop to loop over the stored items

var greet = ["hi" , "bye" , "hello" , "goodbye"];

var length = greet.length;

for (var i = 0 ; i < length ; i++){

	console.log(greet[i]);

} // End of For Loop

// ------------------ Example 3 --------------

var greet = ["hi" , "bye" , "hello" , "goodbye"];

var length = greet.length;

for (var i = 0 ; i < greet.length ; i++){

	console.log(greet[i]);

} // End of For Loop


// Note: You don't need to make a seperate variable
// for the length if you don't want to, and can instead
// just write greet.length when referring to it


		
\end{lstlisting}

\section*{Further Reading}

\href{https://developer.mozilla.org/en-US/docs/Web/JavaScript/Reference/Global_Objects/Array/length}{MDN - length}

\end{document}
