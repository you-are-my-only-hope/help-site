\documentclass[12pt, letterpaper]{article}
\usepackage[utf8]{inputenc}

\usepackage{listings}
\usepackage{color}
\definecolor{lightgray}{rgb}{.9,.9,.9}
\definecolor{gray}{rgb}{.4,.4,.4}
\definecolor{purple}{rgb}{0.65, 0.12, 0.82}

\lstdefinelanguage{JavaScript}{
  keywords={typeof, new, true, false, catch, function, return, null, catch, switch, var, if, in, while, do, else, case, break},
  keywordstyle=\color{blue}\bfseries,
  ndkeywords={class, export, boolean, throw, implements, import, this},
  ndkeywordstyle=\color{darkgray}\bfseries,
  identifierstyle=\color{black},
  sensitive=false,
  comment=[l]{//},
  morecomment=[s]{/*}{*/},
  commentstyle=\color{purple}\ttfamily,
  stringstyle=\color{red}\ttfamily,
  morestring=[b]',
  morestring=[b]"
}

\lstset{
   language=JavaScript,
   backgroundcolor=\color{lightgray},
   extendedchars=true,
   basicstyle=\footnotesize\ttfamily,
   showstringspaces=false,
   showspaces=false,
   numbers=left,
   numberstyle=\footnotesize,
   numbersep=9pt,
   tabsize=2,
   breaklines=true,
   showtabs=false,
   captionpos=b
}

\title{For Loop }
\author{Justin Besteman}
\date{December 30, 2016}

\begin{document}


\maketitle


\section*{Synopsis}

For Loops are great and powerful tools for looping over code for a fixed amount of time.\\

\section*{Why}

A for loop can literally save a coder from writing a 100 lines of code to write or spending hours typing. \\
Remember the array of :

\begin{lstlisting}
//   index    0     1      3        4
var greet = ["Hi","Bye","Hello","Goodbye"];

greet[0]; // "Hi"
greet[1]; // "Bye"
greet[2]; // "Hello"
greet[3]; // "Goodbye"
\end{lstlisting}
Can you imagine if the array is 10, 100, 1000, 10000, or 100000 items in there?
Want happens if for the program, you had to type it all out like it was done in the example of greet \\
Maybe you had to print it to the screen for the user to see. \\
I would ask for a rope and barstool first before doing that.\\
\section*{Syntax}

For Loops start with keyword ``for" and follow by parenthesis\\
It is block level code so(drumroll please) curly braces $\lbrace\rbrace$ \\
Inside the parenthesis can get tricky but stay calm and code\\
for (start ; finish ; counter) \\
This is the general idea with for loops \\
\textbf{Semicolons are used for seperation} not commas\\
\\
Start: Tells the for loop where to start counting\\
Finish: Tells the for loop where to stop or break\\
Counter: Tells the for loop how to count\\

\section*{Advanced}

Generally, a coder will use all three parameters in the for loop (start,finish,counter)\\
But you can omit some and all the parameters if your program calls for it\\
I have been coding for a long time and I can remember on one hand how many times I had to do that 
(and I could lose to three fingers).

\section*{Examples}

\begin{lstlisting}

// ---------------------- Example 1 ----------------

//   index    0     1      2        3
var greet = ["Hi","Bye","Hello","Goodbye"];

greet[0]; // "Hi"
greet[1]; // "Bye"
greet[2]; // "Hello"
greet[3]; // "Goodbye"

// Instead doing this we can do this

for (var i = 0 ; i < 4 ; i++) {

	// The i is going to be changing through every loop
	// First time i = 0
	// Second time i = 1
	// Three time i = 2
	// Four time i = 3
	// Then will break
	// because the middle condition becomes false 
	console.log(greet[i]);

} // End of For Loop I

// Lets breakdown what is in the parenthesis

// var i = 0 - we are saying start counter 0 
// when you are looping through an array you,
// generally want to start at 0 because that is where,
// array will start counting from

// i < 4 - we are saying finish when this statement,
// or condition is true or still true keep going baby!!
// when the middle statement becomes false it will,
// stop or break out of the loop

// i++ - everytime the code loops count by 1,
// saying i++ is the samething as i = i + 1 or i += 1 


// ---------------------- Example 2 ----------------

//   index    0     1      2        3
var greet = ["Hi","Bye","Hello","Goodbye"];

for (var i = 3 ; i >= 0 ; i--){

	// This will console log from the back of the array
	// to the front rather then front to back
	console.log(greet[i]);

} // End of For Loop I

// This for loop counts down instead of counting up
// Notice >= this mean greater or equal too
// <= this less then or equal too


// ---------------------- Example 3 ----------------

//   index    0     1      2        3
var greet = ["Hi","Bye","Hello","Goodbye"];

var start = 0;
var finish = 4;

for (start ; start < finish ; start++){


	console.log(greet[start]);

} // End of For Loop Start


\end{lstlisting}

\end{document}