\documentclass[12pt, letterpaper]{article}
\usepackage[utf8]{inputenc}

\usepackage{listings}
\usepackage{color}
\definecolor{lightgray}{rgb}{.9,.9,.9}
\definecolor{gray}{rgb}{.4,.4,.4}
\definecolor{purple}{rgb}{0.65, 0.12, 0.82}

\lstdefinelanguage{JavaScript}{
  keywords={typeof, new, true, false, catch, function, return, null, catch, switch, var, if, in, while, do, else, case, break},
  keywordstyle=\color{blue}\bfseries,
  ndkeywords={class, export, boolean, throw, implements, import, this},
  ndkeywordstyle=\color{darkgray}\bfseries,
  identifierstyle=\color{black},
  sensitive=false,
  comment=[l]{//},
  morecomment=[s]{/*}{*/},
  commentstyle=\color{purple}\ttfamily,
  stringstyle=\color{red}\ttfamily,
  morestring=[b]',
  morestring=[b]"
}

\lstset{
   language=JavaScript,
   backgroundcolor=\color{lightgray},
   extendedchars=true,
   basicstyle=\footnotesize\ttfamily,
   showstringspaces=false,
   showspaces=false,
   numbers=left,
   numberstyle=\footnotesize,
   numbersep=9pt,
   tabsize=2,
   breaklines=true,
   showtabs=false,
   captionpos=b
}

\title{Switch}
\author{Justin Besteman}
\date{December 30, 2016}

\begin{document}


\maketitle


\section*{Synopsis}

Switch are powerful tool for a coder to test many conditions

\section*{Why}

If a coder is testing only a few conditions then a if, else if, else if are perfect tools to use. \\
But want if there are vast number of things to test. A coder could still use those tools mentioned before \\
but you better get some ice to cool off the fingers when you are done.\\
A switch statement is way to reduce the amount code that needs to be tested

\section*{Syntax}

The switch statement is structure like a train track system where the rails switch (you see what I did there) base on a certian case is achieve. \\
With this tool, you have a switch and cases that will be excuted based on the condition of the switch. \\
After each case, you must use the keywork break for if you don't bad things happen to good people.\\
A switch statement is block lever code so without the break it excuted all the condition after the condition that is met.

\section*{Advanced}

Switch statement have the ability to ``Stack case on top of each other" (My mentor, Sean, taught me that) \\
Meaning you can have more then one condition or case that can be met for that code to excute. \\
At the end of the switch statement, a coder may put the keyword ``default" \\
This will be triggered if none of the condition are met. 

\section*{Examples}

\begin{lstlisting}

// ----------------- Example 1 ------------------

var traffic_light = "green";

switch (traffic_light) {

	case "green":
		console.log("Go");
		break;
	case "yellow":
		console.log("Speed up or slow down");
		break;
	case "red":
		console.log("Stop, Stop, Stoopppp!!!");
		break;
		
} // End of Switch Statement

// ----------------- Example 2 ------------------

var user_pick = 3;

switch (user_pick) {

	case 1:
		console.log("User 1");
		break;
	case 2:
		console.log("User 2");
		break;
	case 3:
		console.log("User 3");
		break;
	case 4:
		console.log("User 4");
		break;
	case 5:
		console.log("User 5");
		break;
		
} // End of Switch Statement

// ----------------- Example 3 ------------------

// This shows how cases can be stacked

var user_pick = 3;

switch (user_pick) {

	case "one":
	case 1:
		console.log("User 1");
		break;
	case "two":
	case 2:
		console.log("User 2");
		break;
	case "three":
	case 3:
		console.log("User 3");
		break;
	case "four":
	case 4:
		console.log("User 4");
		break;
	case "five":
	case 5:
		console.log("User 5");
		break;
		
} // End of Switch Statement

// So this is powerful because if you allow the,
// user enter their choices and if they entered in,
// a integers or string it will still work 

// ----------------- Example 4 ------------------

// This shows the default keyword

var user_pick = 3;

switch (user_pick) {

	case "one":
	case 1:
		console.log("User 1");
		break;
	case "two":
	case 2:
		console.log("User 2");
		break;
	case "three":
	case 3:
		console.log("User 3");
		break;
	case "four":
	case 4:
		console.log("User 4");
		break;
	case "five":
	case 5:
		console.log("User 5");
		break;
	default:
		console.log("Invalid input");
		break;
		
} // End of Switch Statement

// Very awesome here
// With this default statement you test,
// The user input and say if it is not valid

\end{lstlisting}


\end{document}